\documentclass[english]{beamer}

\usepackage{babel}
\usepackage[utf8]{inputenc}
\newcommand{\sectionslide}{\centering\vspace*{25mm}%
  {\bfseries\LARGE \insertsection}}

\usepackage{xmpincl}
\includexmp{CC_Attribution_4.0_International}

\usepackage{csquotes}
\usepackage[backend=biber]{biblatex}
\addbibresource{references.bib}

\author[Arildsen] {
  Thomas Arildsen\\
  {\ttfamily \href{mailto:tha@es.aau.dk}{tha@es.aau.dk}} }

\institute[Dept.\ of Electronic Systems, Aalborg University]
{ Dept.\ of Electronic Systems\\
  Aalborg University}

\usepackage{listings}
\lstdefinelanguage{python}{%
  morekeywords={and, as, assert, break, class, continue, def, del,
    elif, else, except, False, finally, for, from, global, if, import,
    in, is, lambda, None, nonlocal, not, or pass, raise, return, True,
    try, with, while, yield} }
\lstset{
  language=python,
  keywordstyle=\bf\color{blue},
  emph={numpy, matplotlib},
  emphstyle=\bf\color{red},
  numbers=left,
  numberstyle=\footnotesize,
  basicstyle=\footnotesize\ttfamily,
  showstringspaces=false,
  backgroundcolor=\color{gray!30},
  aboveskip=4pt,
  frame=leftline
}

\title{Scientific software development best practices}

\subject{Scientific software development best practices}

\begin{document}

%%%%%%%%%%%%%%%%%%%%%%%%%%%%%%%%%%%%%%%%%%%%%%%%%%%%%%%%%%
%%%%%%%%%%%%%%%%%%%%%%%%%%%%%%%%%%%%%%%%%%%%%%%%%%%%%%%%%%
%%%%%%%%%%%%%%%%%%%%%%%%%%%%%%%%%%%%%%%%%%%%%%%%%%%%%%%%%%

\begin{frame}[plain]
  \titlepage
\end{frame}

%%%%%%%%%%%%%%%%%%%%%%%%%%%%%%%%%%%%%%%%%%%%%%%%%%%%%%%%%%

\begin{frame}[plain]
  \begin{figure}
    \centering
    \includegraphics[width=4cm]{cc}
    \hspace{1cm}
    \includegraphics[width=4cm]{by}
  \end{figure}
  Best Practices for Scientific Computing by Thomas Arildsen is
  licensed under a
  \href{http://creativecommons.org/licenses/by/4.0/}{Creative Commons
    Attribution 4.0 International License}.  Based on a work at
  \href{http://dx.doi.org/10.1371/journal.pbio.1001745}{DOI:
    10.1371/journal.pbio.1001745} \cite{Wilson2013}.
\end{frame}

%%%%%%%%%%%%%%%%%%%%%%%%%%%%%%%%%%%%%%%%%%%%%%%%%%%%%%%%%%

\begin{frame}{Scientific software development best practices}{Agenda}
  \tableofcontents
\end{frame}

%%%%%%%%%%%%%%%%%%%%%%%%%%%%%%%%%%%%%%%%%%%%%%%%%%%%%%%%%%
%%%%%%%%%%%%%%%%%%%%%%%%%%%%%%%%%%%%%%%%%%%%%%%%%%%%%%%%%%
%%%%%%%%%%%%%%%%%%%%%%%%%%%%%%%%%%%%%%%%%%%%%%%%%%%%%%%%%%

\section{Introduction}

\begin{frame}
  % 
  \sectionslide
\end{frame}

%%%%%%%%%%%%%%%%%%%%%%%%%%%%%%%%%%%%%%%%%%%%%%%%%%%%%%%%%%

\begin{frame}{\insertsection}{Scientific software development}

\begin{itemize}
\item When we code models, simulation scripts, and scientific
  computing in general, we should not just sit down in front of our
  computers and start typing as we begin to consider what the program
  should do.
\item Going about this in a sensible order and with some structure
  around the process is necessary to avoid serious mistakes that will
  cost us a lot of time and energy to fix afterwards.
\item Numerous programmers, software engineers and scientists before
  us have made countless mistakes that we can learn from and at least
  try to do it ``right'' from the beginning.
\end{itemize}

\end{frame}

%%%%%%%%%%%%%%%%%%%%%%%%%%%%%%%%%%%%%%%%%%%%%%%%%%%%%%%%%%

\begin{frame}{\insertsection}{Intended learning outcomes}
  %
  After today's lecture you will have knowledge about:
  \begin{itemize}
  \item Some ``best practice'' guidelines for developing software.
  \item General advice for developing software with a scientific
    software angle.
  \item Principles of testing and validation of scientific software.
  \end{itemize}
  You will be able to:
  \begin{itemize}
  \item Apply a set of guidelines to the development of software in
    your projects.
  \item Structure your software development process for clarity and
    reliability of the code.
  \end{itemize}
\end{frame}

%%%%%%%%%%%%%%%%%%%%%%%%%%%%%%%%%%%%%%%%%%%%%%%%%%%%%%%%%%

\begin{frame}{\insertsection}{Source}
  %
  These slides are based on and quote from:
  \begin{center}
    \printbibliography
  \end{center}
\end{frame}

%%%%%%%%%%%%%%%%%%%%%%%%%%%%%%%%%%%%%%%%%%%%%%%%%%%%%%%%%%


\section{Scientific Software Development}

\begin{frame}
  % 
  \sectionslide
\end{frame}

%%%%%%%%%%%%%%%%%%%%%%%%%%%%%%%%%%%%%%%%%%%%%%%%%%%%%%%%%%

\begin{frame}{\insertsection}{What do we mean by that?}
  %
  \begin{block}{Scientific software vs. ``other software''}
    \begin{itemize}
    \item When you read about software development practices, these
      are often concerned with software that is aimed at helping users
      do something (word processing, spreadsheets, web browsing etc.,
      etc.).
    \item Or, maybe we are talking about \emph{embedded software} --
      software that is not directly visible to a user but typically
      plays a more technical role of making some technological
      ``apparatus'' work.
    \item What is scientific software?
    \end{itemize}
  \end{block}
\end{frame}

%%%%%%%%%%%%%%%%%%%%%%%%%%%%%%%%%%%%%%%%%%%%%%%%%%%%%%%%%%

\begin{frame}{\insertsection}
  %
  \begin{block}{Scientific software}
    \begin{itemize}
    \item Often written by scientists/researchers for their own use.
    \item Not designed to satisfy detailed requirements of human
      users.
    \item Mainly intended to answer one or more fairly specific
      scientific, often mathematically formulated questions.
    \item Should support reproducibility of results.
    \item Validation important.
    \end{itemize}
  \end{block}
\end{frame}

%%%%%%%%%%%%%%%%%%%%%%%%%%%%%%%%%%%%%%%%%%%%%%%%%%%%%%%%%%

\begin{frame}{\insertsection}
  %
  \begin{itemize}
  \item Several of these slides' pieces of advice are not aimed at
    scientific software \emph{development} as such.
  \item Rather, they are more generally aimed at making science
    \emph{reproducible}.
  \item However, this is also a very useful practice for us to learn.
  \end{itemize}
\end{frame}

%%%%%%%%%%%%%%%%%%%%%%%%%%%%%%%%%%%%%%%%%%%%%%%%%%%%%%%%%%

\section{Write Programs for People}

\begin{frame}
  % 
  \sectionslide
\end{frame}

%%%%%%%%%%%%%%%%%%%%%%%%%%%%%%%%%%%%%%%%%%%%%%%%%%%%%%%%%%

\begin{frame}{\insertsection}
  % 
  \begin{quote}
    Any fool can write code that a computer can understand. Good
    programmers write code that humans can understand.\\
    \hfill -- Martin Fowler, 1999
  \end{quote}
  We need to write software that both executes correctly \emph{and} is
  readable and understandable to others.
  \begin{itemize}
  \item If not, it becomes difficult for others (and \alert{you}) to
    understand what the program does.
  \item Also makes it difficult to check whether the program does it
    correctly.
  \end{itemize}
\end{frame}

%%%%%%%%%%%%%%%%%%%%%%%%%%%%%%%%%%%%%%%%%%%%%%%%%%%%%%%%%%

\begin{frame}{\insertsection}{Humans are simple-minded\ldots}
  % 
  A program should not require its readers to hold more than a handful
  of facts in memory at once.~\cite{Wilson2013}.
  \begin{itemize}
  \item We can only hold about a handful of items in ``working
    memory'' at a time.
  \item ``Items'' can be a single fact or some chunk of facts.
  \item Take this principle into account by dividing our programs
    logically into functions, each of which carries out a single task.
  \item Functions -- and the tasks they carry out -- should be easily
    understandable.
  \end{itemize}
\end{frame}

%%%%%%%%%%%%%%%%%%%%%%%%%%%%%%%%%%%%%%%%%%%%%%%%%%%%%%%%%%

\begin{frame}[fragile]{\insertsection}{Humans are simple-minded\ldots}
  % 
  Example:
  \begin{lstlisting}
def line_slope(x1, y1, x2, y2):
    # Some calculation
  \end{lstlisting}
  vs.
  \begin{lstlisting}
def line_slope(point1, point2):
    # Some calculation
  \end{lstlisting}
\end{frame}

%%%%%%%%%%%%%%%%%%%%%%%%%%%%%%%%%%%%%%%%%%%%%%%%%%%%%%%%%%

\begin{frame}{\insertsection}{Naming conventions}
  % 
  Make names consistent, distinctive, and
  meaningful.~\cite{Wilson2013}.
  \begin{itemize}
  \item It makes the code easier to understand when the reader does
    not have to switch between different naming conventions through
    the document.
  \item Use names that describe what the given variable or function
    is.
  \item For example:
    \begin{itemize}
    \item Not just \texttt{a} or \texttt{foo}.
    \item Not too similar -- \texttt{result1} and \texttt{result2}.
    \end{itemize}
  \item On the other hand, if a function's documentation clearly
    states that its purpose is to evaluate the polynomial $f(x) = ax^2
    + bx + c$, maybe \texttt{a}, \texttt{b}, and \texttt{c} are not
    such a bad idea\ldots
  \end{itemize}
\end{frame}

%%%%%%%%%%%%%%%%%%%%%%%%%%%%%%%%%%%%%%%%%%%%%%%%%%%%%%%%%%

\begin{frame}[fragile]{\insertsection}{Style and formatting}
  % 
  Make code style and formatting consistent.~\cite{Wilson2013}.
  \begin{itemize}
  \item Do not mix different case styles.
  \item For example:
    \begin{itemize}
    \item Do not use both \texttt{CamelCase} and
      \texttt{pothole\textunderscore{}case}.
    \end{itemize}
  \item In Python, for example, do not mix tabs and spaces.
  \item Keep indentation levels consistent.
  \end{itemize}
  \onslide<2->
  Example:
  \lstinputlisting[lastline=9]{examples/spacing.py}
\end{frame}

%%%%%%%%%%%%%%%%%%%%%%%%%%%%%%%%%%%%%%%%%%%%%%%%%%%%%%%%%%

\section{Let the Computer Do the Work}

\begin{frame}
  % 
  \sectionslide
\end{frame}

%%%%%%%%%%%%%%%%%%%%%%%%%%%%%%%%%%%%%%%%%%%%%%%%%%%%%%%%%%

\begin{frame}{\insertsection}{The computer is your slave\ldots}
  % 
  Make the computer repeat tasks and save recent commands in a file
  for re-use.~\cite{Wilson2013}.
  \begin{itemize}
  \item If we repeat things over and over again, such as processing
    large numbers of files the same way or regenerating figures for
    new data, we waste time and are bound to make mistakes at some
    point.
  \item In Python, for example, script tasks to avoid typing things
    over and over.
  \item In command-line interfaces, use command history to recall
    recent commands and only change the necessary parameters.

    (Also available for commands typed manually in IPython).
  \item Generally, script it if you can in preference to recalling and
    editing commands manually.
  \end{itemize}
\end{frame}

%%%%%%%%%%%%%%%%%%%%%%%%%%%%%%%%%%%%%%%%%%%%%%%%%%%%%%%%%%

\begin{frame}{\insertsection}{Workflows}
  % 
  \vspace{-2mm}
  Use a build tool to automate workflows.~\cite{Wilson2013}.
  \begin{itemize}
  \item Particularly for compiled languages, it is common to use a
    \emph{build tool} to compile and link program files, i.e. keeping
    track of compiling and linking only the necessary files.
  \item Examples: \emph{Make} -- a classic for C/C++ etc., \emph{ant}
    -- particularly used for Java.
  \item These tools can also be used for more general tasks than
    compiling and linking.
  \item We can use them for our ``computing'' tasks, such as:

    ``To produce figure 4, we need to make sure that \texttt{program1}
    and \texttt{program2} have been run and that \texttt{analysis} has
    been run afterwards to combine their data before we finally run
    \texttt{generate\textunderscore{}figure}.''
  \item For Python, see \url{http://paver.github.io/paver/},
    \url{http://www.scons.org/}.
  \end{itemize}
\end{frame}

%%%%%%%%%%%%%%%%%%%%%%%%%%%%%%%%%%%%%%%%%%%%%%%%%%%%%%%%%%

\section{Make Incremental Changes}

\begin{frame}
  % 
  \sectionslide
\end{frame}

%%%%%%%%%%%%%%%%%%%%%%%%%%%%%%%%%%%%%%%%%%%%%%%%%%%%%%%%%%

\begin{frame}{\insertsection}{``Agile'' development}
  % 
  Work in small steps with frequent feedback and course
  correction.~\cite{Wilson2013}.
  \begin{itemize}
  \item We may not know from the beginning, exactly all of the steps
    that the program must go through (particularly true for
    scientific software development).
  \item Better to work in small steps than trying to plan everything
    months or years ahead.
  \item Work in steps of about an hour grouped into iterations of
    about a weeks duration.
  \item Accommodates cognitive constraints of (human) programmers.
  \end{itemize}
\end{frame}

%%%%%%%%%%%%%%%%%%%%%%%%%%%%%%%%%%%%%%%%%%%%%%%%%%%%%%%%%%

\begin{frame}{\insertsection}{Versioning}
  % 
  Use a version control system.~\cite{Wilson2013}.
  \begin{itemize}
  \item A version control system (VCS) is a program that stores and
    keeps track of different \emph{versions} of files in a
    \emph{repository}.
  \item It is typically used for software development, i.e. storing
    versions of source code files etc., but it can be applied much
    more generally to all types of files.
  \item Working with a VCS is often based on \emph{differences}
    between versions of the individual files.
  \item Ideally for text-based files. Not so well-suited for binary
    files.
  \item Can be used for example for (LaTeX) report documents as well
    as your source code.
  \item VCS examples: Subversion, git, Mercurial.
  \end{itemize}
\end{frame}

%%%%%%%%%%%%%%%%%%%%%%%%%%%%%%%%%%%%%%%%%%%%%%%%%%%%%%%%%%

\begin{frame}{\insertsection}{Versioning}
  % 
  Put everything that has been created manually in version
  control.~\cite{Wilson2013}.
  \begin{itemize}
  \item For example:
    \begin{itemize}
    \item Program source code.
    \item Report documents.
    \item Working notes.
    \end{itemize}
  \item Files that can be derived automatically from the above are not
    necessary to track in the VCS (maybe in some cases for
    convenience).
  \item Especially in the case of large binary files, it might be a
    good idea to store them externally but store meta-data about them
    in the VCS.
    \begin{itemize}
    \item git has add-ons for this sort of thing: git-media,
      git-annex, git-submodules.
    \end{itemize}
  \end{itemize}
\end{frame}

%%%%%%%%%%%%%%%%%%%%%%%%%%%%%%%%%%%%%%%%%%%%%%%%%%%%%%%%%%

\section{Don't Repeat Yourself}

\begin{frame}
  % 
  \sectionslide
\end{frame}

%%%%%%%%%%%%%%%%%%%%%%%%%%%%%%%%%%%%%%%%%%%%%%%%%%%%%%%%%%

\begin{frame}{\insertsection}{Uniqueness}
  % 
  Every piece of data must have a single authoritative representation
  in the system.~\cite{Wilson2013}.
  \begin{itemize}
  \item Anything that is duplicated in two or more places is difficult
    to keep track of.
  \item Sooner or later, we are going to forget to update one of these
    representations of the data or code, and our data or program
    becomes inconsistent.
  \item For example, physical constants used in a program should be
    define in only one place so it is unambiguous which definition is
    being used by the program.
  \item Raw data files (e.g. images, audio, measurements of some kind)
    should exist in only one copy.
  \end{itemize}
\end{frame}

%%%%%%%%%%%%%%%%%%%%%%%%%%%%%%%%%%%%%%%%%%%%%%%%%%%%%%%%%%

\begin{frame}{\insertsection}{Modularity}
  % 
  Modularize code rather than copying and pasting.~\cite{Wilson2013}.

  (Small scale, own code)
  \begin{itemize}
  \item If a certain functionality is needed several places in a
    program, modularize it by defining it as a function which is
    called each of these places.
  \item Cloning code and writing this functionality into the source
    code at different places in the program introduces several places
    we need to maintain when for example correcting bugs etc.
  \item Also helps people remember the functionality as one single
    ``chunk'', making it easier to keep track of.
  \item Modularized code is easier to re-use elsewhere.
  \end{itemize}
\end{frame}

%%%%%%%%%%%%%%%%%%%%%%%%%%%%%%%%%%%%%%%%%%%%%%%%%%%%%%%%%%

\begin{frame}{\insertsection}{Re-use}
  % 
  Re-use code instead of rewriting it.~\cite{Wilson2013}.

  (Large scale)
  \begin{itemize}
  \item If you need particular functionality in your program that
    others have programmed before you, use available code instead of
    programming it yourself.
  \item Huge amounts of software that might solve just your problem
    are available as free, open-source software online.
  \item You probably cannot program it more efficiently yourself
    anyway.
  \item For example, we utilise lots of functionality from NumPy and
    Matplotlib instead of implementing it ourselves.
    \begin{overlayarea}{\textwidth}{2cm}
      \only<1>\ldots
      \only<2>{
      \item That being said, for educational purposes it can be a good
        idea to program certain project-relevant functionality
        yourselves as it helps you understand it better.}
    \end{overlayarea}
  \end{itemize}
\end{frame}

%%%%%%%%%%%%%%%%%%%%%%%%%%%%%%%%%%%%%%%%%%%%%%%%%%%%%%%%%%

\section{Plan for Mistakes}

\begin{frame}
  % 
  \sectionslide
\end{frame}

%%%%%%%%%%%%%%%%%%%%%%%%%%%%%%%%%%%%%%%%%%%%%%%%%%%%%%%%%%

\begin{frame}[fragile]{\insertsection}{S\ldots\ happens\ldots}
  % 
  Add assertions to programs to check their
  operation.~\cite{Wilson2013}.
  \begin{itemize}
  \item An assertion is a conditional statement of something we know
    must be true.
  \item We add the statement as a ``sanity check''.
  \item If the statement fails, we know something is wrong.
  \item Example:
    \lstinputlisting{examples/assertion.py}
  \item Serve to catch errors as soon as possible -- simplifies
    debugging.
  \item Also serve as documentation -- help explain how the program
    works.
  \end{itemize}
\end{frame}

%%%%%%%%%%%%%%%%%%%%%%%%%%%%%%%%%%%%%%%%%%%%%%%%%%%%%%%%%%

\begin{frame}{\insertsection}{Testing}
  % 
  \vspace{-2mm}
  Use an off-the-shelf unit testing library.~\cite{Wilson2013}.
  \begin{itemize}
  \item Testing is important to make sure that the program works
    correctly.
  \item Unit tests test whether a single part of the software, for
    example a function, works as intended.
  \item Integration tests test whether different parts of the code
    work correctly when used together.
  \item For example, if we have implemented a function to numerically
    evaluate some function, maybe we can analytically determine the
    correct value in some selected points.

    A test can then be to run the function on these points and compare
    the returned values to the analytically calculated values.
  \item In Python, we can use the module \texttt{unittest} to automate
    this kind of testing.
  \end{itemize}
\end{frame}

%%%%%%%%%%%%%%%%%%%%%%%%%%%%%%%%%%%%%%%%%%%%%%%%%%%%%%%%%%

\begin{frame}{\insertsection}{Testing and debugging}
  % 
  Turn bugs into test cases.~\cite{Wilson2013}.
  \begin{itemize}
  \item When we find bugs in our programs (and we will\ldots), we
    should turn those bugs into tests with code that trigger this bug.
  \item This prevents this (kind of) bug from re-appearing at a later
    point in time and/or at another place in the program.
  \item Proper testing improves confidence in our program.
  \item Also encourages programmers to write code that is easier to
    test. This way it usually becomes easier to understand as well.
  \end{itemize}
\end{frame}

%%%%%%%%%%%%%%%%%%%%%%%%%%%%%%%%%%%%%%%%%%%%%%%%%%%%%%%%%%

\begin{frame}{\insertsection}{Debugging}
  % 
  Use a symbolic debugger.~\cite{Wilson2013}.
  \begin{itemize}
  \item You may be able to debug your program using for example
    \texttt{print} statements here and there to watch the values of
    certain variables.
  \item This is time-consuming and it can be very tricky to track down
    the source of a problem.
  \item Using a so-called symbolic debugger lets you inspect your
    program at run-time -- much more efficient.
  \item You can set break-points where the debugger stops the program
    and you can inspect the values of different variables at that
    point in the program.
  \item You can also step through the code line by line and observe
    what happens in each line.
  \end{itemize}
\end{frame}

%%%%%%%%%%%%%%%%%%%%%%%%%%%%%%%%%%%%%%%%%%%%%%%%%%%%%%%%%%

\section{Optimize Software Only after It Works}% Correctly}

\begin{frame}
  % 
  \sectionslide
\end{frame}

%%%%%%%%%%%%%%%%%%%%%%%%%%%%%%%%%%%%%%%%%%%%%%%%%%%%%%%%%%

\begin{frame}{\insertsection}
  % 
  \begin{quote}
    Premature optimization is the root of all evil.\\
    \hfill -- Donald Knuth, 1974
  \end{quote}
  \begin{itemize}
  \item The father of \TeX\ once said this and these words contain a
    lot of useful wisdom\ldots
  \item Once our program works correctly (which we can identify
    through testing), we can start looking at how to make it faster,
    use less memory, or maybe just more easily understandable.
  \end{itemize}
\end{frame}

%%%%%%%%%%%%%%%%%%%%%%%%%%%%%%%%%%%%%%%%%%%%%%%%%%%%%%%%%%

\begin{frame}{\insertsection}{Profiling}
  % 
  Use a profiler to identify bottlenecks.~\cite{Wilson2013}.
  \begin{itemize}
  \item In many cases, we cannot predict correctly which parts of our
    program take up the largest amount of time.
  \item A profiler tool analyses our program at run-time and records
    how much time the program spends in each line or function of the
    program.
  \item This allows us to identify which parts of the program are
    particularly heavy (consume the largest fractions of the run
    time).
  \item Having identified these ``bottlenecks'', we can then
    efficiently concentrate on optimizing the one or few places where
    it really makes sense.
  \item It usually only makes sense to optimize a few selected parts
    of the code.
  \end{itemize}
\end{frame}

%%%%%%%%%%%%%%%%%%%%%%%%%%%%%%%%%%%%%%%%%%%%%%%%%%%%%%%%%%

\begin{frame}{\insertsection}{Programming language}
  % 
  Write code in the highest-level language
  possible.~\cite{Wilson2013}.
  \begin{itemize}
  \item Programmers tend to produce the same amount of lines of code
    per time regardless of the programming language.
  \item Using a higher-level programming language, you can usually do
    more with less code and so, programming is more efficient here.
  \item Then we can, if necessary, optimise specific parts of the code
    in a lower-level language later on.
  \item In both of the above aspects, Python is an excellent choice of
    language.
    \begin{itemize}
    \item It is high-level and thus efficient to program in.
    \item It makes it easy to integrate code from other, lower-level
      languages such as C or Fortran if needed.
    \end{itemize}
  \end{itemize}
\end{frame}

%%%%%%%%%%%%%%%%%%%%%%%%%%%%%%%%%%%%%%%%%%%%%%%%%%%%%%%%%%

\section{Document Design and Purpose}

\begin{frame}
  % 
  \sectionslide
\end{frame}

%%%%%%%%%%%%%%%%%%%%%%%%%%%%%%%%%%%%%%%%%%%%%%%%%%%%%%%%%%

\begin{frame}[fragile]{\insertsection}{Documentation}
  % 
  Document interfaces and reasons, not
  implementations.~\cite{Wilson2013}.
  \begin{itemize}
  \item It is important to keep code well-documented to make it easier
    for the reader to understand and for others to maintain later on.
  \item Documentation on design decisions, i.e. why things were done
    the way they were done are useful.
  \item Documentation of interfaces is useful. For example
    documentation at the beginning of a function that explains what
    the function does, which arguments it takes, and what it returns.
  \item Documentation of the detailed mechanics in the code is often
    not very useful:
    \begin{lstlisting}[numbers=none]
i = i + 1   # Increment variable 'i' by one
    \end{lstlisting}
  \end{itemize}
\end{frame}

%%%%%%%%%%%%%%%%%%%%%%%%%%%%%%%%%%%%%%%%%%%%%%%%%%%%%%%%%%

\begin{frame}{\insertsection}{Refactoring}
  % 
  Refactor code in preference to explaining how it
  works.~\cite{Wilson2013}.
  \begin{itemize}
  \item In some cases, particular parts of code may need unusually
    long and complicated descriptions.
  \item In such cases it is often better to re-write that part of the
    code.
  \item Known as \emph{refactoring} the code: changing the code such
    that it still produces the same results.
  \item Refactor difficult-to-document parts of the code to make them
    easier to understand.
  \item Not always possible -- some things just are complicated with
    no simpler way to do them.
  \end{itemize}
\end{frame}

%%%%%%%%%%%%%%%%%%%%%%%%%%%%%%%%%%%%%%%%%%%%%%%%%%%%%%%%%%

\begin{frame}{\insertsection}{Embed documentation}
  % 
  Embed the documentation for a piece of software in that
  software.~\cite{Wilson2013}.
  \begin{itemize}
  \item The documentation of a program should be embedded within the
    code itself rather than in an external document.
  \item Makes it easier to read the documentation while inspecting the
    code.
  \item Increases the probability that the documentation gets updated
    accordingly when the code is changed.
  \item If we want external, nicely-formatted documentation as well,
    this can be generated from the embedded documentation. For example
    using Sphinx or Dexy.
  \end{itemize}
\end{frame}

%%%%%%%%%%%%%%%%%%%%%%%%%%%%%%%%%%%%%%%%%%%%%%%%%%%%%%%%%%

\section{Collaborate}

\begin{frame}
  % 
  \sectionslide
\end{frame}

%%%%%%%%%%%%%%%%%%%%%%%%%%%%%%%%%%%%%%%%%%%%%%%%%%%%%%%%%%

\begin{frame}{\insertsection}{Review}
  % 
  Use pre-merge code reviews.~\cite{Wilson2013}.
  \begin{itemize}
  \item Code review is having fellow programmers read through your
    code.
  \item Code reviews are a good way to assure the quality of the code
    you produce -- catch bugs and improve readability.
  \item Also helps spread knowledge of how the code works to other
    team members.
  \item Pre-merge code review, as suggested here, means that the code
    should be reviewed before being checked into the VCS repository.
  \item The argument for this here is that if reviews do not have to
    be done before being checked in, they will probably not get
    reviewed at all.
  \end{itemize}
\end{frame}

%%%%%%%%%%%%%%%%%%%%%%%%%%%%%%%%%%%%%%%%%%%%%%%%%%%%%%%%%%

\begin{frame}{\insertsection}{Pair programming}
  % 
  Use pair programming when bringing someone new up to speed and when
  tackling particularly tricky problems.~\cite{Wilson2013}.
  \begin{itemize}
  \item Pair programming is a fairly extreme form of code review.
  \item One person (``driver'') writes the actual code while the other
    (``navigator'') is watching, providing comments on the way the
    code is implemented.
  \item The navigator can maintain a better overview while the driver
    is ``buried'' in the details.
  \item This can be quite intrusive (similar to having a passenger
    constantly commenting on how you drive your car).
  \item Therefore the recommendation here is to use it when
    introducing someone new to the code or when dealing with
    particularly difficult problems.
  \end{itemize}
\end{frame}

%%%%%%%%%%%%%%%%%%%%%%%%%%%%%%%%%%%%%%%%%%%%%%%%%%%%%%%%%%

\begin{frame}{\insertsection}{``Issues''}
  % 
  Use an issue tracking tool.~\cite{Wilson2013}.
  \begin{itemize}
  \item This is a tool to keep track of bugs in the program, what
    needs to be programmed, what needs to be reviewed etc.
  \item A sort of to-do list kept together with the code.
  \item Helps provide all team members with a joint overview of the
    project.
  \item Usually integrated with the version control.
  \item Comes integrated with several free online VCS repository
    platforms such as Bitbucket and GitHub or in the form of locally
    installed tools such as Trac.
  \end{itemize}
\end{frame}

%%%%%%%%%%%%%%%%%%%%%%%%%%%%%%%%%%%%%%%%%%%%%%%%%%%%%%%%%%

\section{Summary}

\begin{frame}
  % 
  \sectionslide
\end{frame}

%%%%%%%%%%%%%%%%%%%%%%%%%%%%%%%%%%%%%%%%%%%%%%%%%%%%%%%%%%

\begin{frame}{\insertsection}
  % 
  Today we have gone through a number of general recommendations for
  best practices in scientific software development:
  \begin{itemize}
  \item Scientific Software Development
  \item Write Programs for People
  \item Let the Computer Do the Work
  \item Make Incremental Changes
  \item Don’t Repeat Yourself
  \item Plan for Mistakes
  \item Optimize Software Only after It Works
  \item Document Design and Purpose
  \item Collaborate
  \end{itemize}
\end{frame}

%%%%%%%%%%%%%%%%%%%%%%%%%%%%%%%%%%%%%%%%%%%%%%%%%%%%%%%%%%

\end{document}
